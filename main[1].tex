\documentclass[journal,12pt,twocolumn]{IEEEtran}

\usepackage{setspace}
\usepackage{gensymb}

\singlespacing


\usepackage[cmex10]{amsmath}

\usepackage{amsthm}

\usepackage{mathrsfs}
\usepackage{txfonts}
\usepackage{stfloats}
\usepackage{bm}
\usepackage{cite}
\usepackage{cases}
\usepackage{subfig}

\usepackage{longtable}
\usepackage{multirow}

\usepackage{enumitem}
\usepackage{mathtools}
\usepackage{steinmetz}
\usepackage{tikz}
\usepackage{circuitikz}
\usepackage{verbatim}
\usepackage{tfrupee}
\usepackage[breaklinks=true]{hyperref}
\usepackage{graphicx}
\usepackage{tkz-euclide}
\usepackage{float}

\usetikzlibrary{calc,math}
\usepackage{listings}
\usepackage{color} %%
\usepackage{array} %%
\usepackage{longtable} %%
\usepackage{calc} %%
\usepackage{multirow} %%
\usepackage{hhline} %%
\usepackage{ifthen} %%
\usepackage{lscape}
\usepackage{multicol}
\usepackage{chngcntr}

\DeclareMathOperator*{\Res}{Res}

\renewcommand\thesection{\arabic{section}}
\renewcommand\thesubsection{\thesection.\arabic{subsection}}
\renewcommand\thesubsubsection{\thesubsection.\arabic{subsubsection}}

\renewcommand\thesectiondis{\arabic{section}}
\renewcommand\thesubsectiondis{\thesectiondis.\arabic{subsection}}
\renewcommand\thesubsubsectiondis{\thesubsectiondis.\arabic{subsubsection}}


\hyphenation{op-tical net-works semi-conduc-tor}
\def\inputGnumericTable{} %%

\lstset{
%language=C,
frame=single,
breaklines=true,
columns=fullflexible
}
\begin{document}


\newtheorem{theorem}{Theorem}[section]
\newtheorem{problem}{Problem}
\newtheorem{proposition}{Proposition}[section]
\newtheorem{lemma}{Lemma}[section]
\newtheorem{corollary}[theorem]{Corollary}
\newtheorem{example}{Example}[section]
\newtheorem{definition}[problem]{Definition}

\newcommand{\BEQA}{\begin{eqnarray}}
\newcommand{\EEQA}{\end{eqnarray}}
\newcommand{\define}{\stackrel{\triangle}{=}}
\bibliographystyle{IEEEtran}
\providecommand{\mbf}{\mathbf}
\providecommand{\pr}[1]{\ensuremath{\Pr\left(#1\right)}}
\providecommand{\qfunc}[1]{\ensuremath{Q\left(#1\right)}}
\providecommand{\sbrak}[1]{\ensuremath{{}\left[#1\right]}}
\providecommand{\lsbrak}[1]{\ensuremath{{}\left[#1\right.}}
\providecommand{\rsbrak}[1]{\ensuremath{{}\left.#1\right]}}
\providecommand{\brak}[1]{\ensuremath{\left(#1\right)}}
\providecommand{\lbrak}[1]{\ensuremath{\left(#1\right.}}
\providecommand{\rbrak}[1]{\ensuremath{\left.#1\right)}}
\providecommand{\cbrak}[1]{\ensuremath{\left\{#1\right\}}}
\providecommand{\lcbrak}[1]{\ensuremath{\left\{#1\right.}}
\providecommand{\rcbrak}[1]{\ensuremath{\left.#1\right\}}}
\theoremstyle{remark}
\newtheorem{rem}{Remark}
\newcommand{\sgn}{\mathop{\mathrm{sgn}}}
\providecommand{\abs}[1]{\left\vert#1\right\vert}
\providecommand{\res}[1]{\Res\displaylimits_{#1}}
\providecommand{\norm}[1]{\left\lVert#1\right\rVert}
%\providecommand{\norm}[1]{\lVert#1\rVert}
\providecommand{\mtx}[1]{\mathbf{#1}}
\providecommand{\mean}[1]{E\left[ #1 \right]}
\providecommand{\fourier}{\overset{\mathcal{F}}{ \rightleftharpoons}}
%\providecommand{\hilbert}{\overset{\mathcal{H}}{ \rightleftharpoons}}
\providecommand{\system}{\overset{\mathcal{H}}{ \longleftrightarrow}}
%\newcommand{\solution}[2]{\textbf{Solution:}{#1}}
\newcommand{\solution}{\noindent \textbf{Solution: }}
\newcommand{\cosec}{\,\text{cosec}\,}
\providecommand{\dec}[2]{\ensuremath{\overset{#1}{\underset{#2}{\gtrless}}}}
\newcommand{\myvec}[1]{\ensuremath{\begin{pmatrix}#1\end{pmatrix}}}
\newcommand{\mydet}[1]{\ensuremath{\begin{vmatrix}#1\end{vmatrix}}}
\numberwithin{equation}{subsection}
\makeatletter
\@addtoreset{figure}{problem}
\makeatother
\let\StandardTheFigure\thefigure
\let\vec\mathbf
\renewcommand{\thefigure}{\theproblem}
\def\putbox#1#2#3{\makebox[0in][l]{\makebox[#1][l]{}\raisebox{\baselineskip}[0in][0in]{\raisebox{#2}[0in][0in]{#3}}}}
\def\rightbox#1{\makebox[0in][r]{#1}}
\def\centbox#1{\makebox[0in]{#1}}
\def\topbox#1{\raisebox{-\baselineskip}[0in][0in]{#1}}
\def\midbox#1{\raisebox{-0.5\baselineskip}[0in][0in]{#1}}
\vspace{3cm}
\title{Assignment No.6}
\author{Mr.Ganesh Yadav}
\maketitle
\newpage
\bigskip
\renewcommand{\thefigure}{\theenumi}
\renewcommand{\thetable}{\theenumi}
Download all python codes from
\begin{lstlisting}
svn co https://github.com/Ganeshyadav712/Assignment-7.git
\end{lstlisting}
%
Question taken from
\begin{lstlisting}
https://github.com/gadepall/ncert/blob/main/linalg/optimization/gvv_ncert_opt.pdf question 2.7
\end{lstlisting}
\section{Question No 1}
A window is in the form of a rectangle surmounted by a semi-circular opening. The total perimeter of the window is 10 metres. Find the dimensions of the rectangle so as to admit maximum light through the whole opening.
\section{Solution}
Let x and y be the length and width of rectangle part of window respectively.\\
Let A be the opening area of the window which admits Light. Obviously, for admitting the maximum light through the opening, A must be maximum.\\
Now \\A = Area of rectangle + Area of the semi-circle.
\begin{align}
    \vec A&= xy+\frac{1}{2}\pi.\frac{x^2}{4}\\
    \vec A&= xy+\frac{\pi.x^2}{8}\\
     \vec A&= x{\frac{5-x(\pi +2)}{4}}+\frac{\pi.x^2}{8}\\
     \vec A&= 5x -{\frac{(\pi +2)x^2}{4}}+\frac{\pi.x^2}{8}\\
     \vec A&= 5x -{\frac{(\pi +2)}{4}}-\frac{\pi}{8}x^2\\
     \vec A&= 5x -{\frac{(\pi +4)}{8}}x^2\\
     \frac{dA}{dX}&=5-\frac{\pi +4}{8}2X\\
     \end{align}
     from question
\begin{align}
\therefore x+2y+\pi \frac{x}{2}&=10\\
    x\frac{\pi}{2}+1+2y&=0\\
    2y&=10-x\frac{\pi +2}{2}\\
    y&=5-\frac{x(\pi +2)}{4}
\end{align}
for maximum or minimum value of \vec A,
\begin{align}
 \dfrac{dA}{dX}=0\\
 5- \dfrac{\pi+4}{8}.2X=0\\
    \vec X=\dfrac{20}{\pi+4}\\
    \frac{d^2A}{dx^2}\Bigg] \vec X=\dfrac{20}{\pi+4}
\end{align}
for $\vec X=\dfrac{20}{\pi+4}, \vec A$ is maximum\\
and hence \begin{align}
    \Vec{Y}=5-\dfrac{20}{\pi+4}.\dfrac{\pi+2}{4}\\
    5-\dfrac{5(\pi+2)}{\pi+4}\\
    \dfrac{5\pi+20-5\pi-10}{\pi+4}\\
    \dfrac{10}{\pi+4}
    \end{align} 
    $\therefore$  
    for maximum $\vec A$ for admitting the maximum light \\
    length of rectangle = $\vec X$= $\dfrac{20}{\pi+4}\\$
    breadth of rectangle = $\vec Y$= \dfrac{10}{\pi+4}    
    \end{document}
